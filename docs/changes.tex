\section{Modifications}
The following changes were made to existing constructs of the AST.\\
Each comes with an explanation of why it was deemed necessary in the context of other additions to the language.

\subsection{Assignments}
\t{SET} replaces both \t{SET\_VAR} and \t{SET\_ARRAY} as well as what was for some time \t{SET\_DEREF}.\\
\begin{itemize}
    \item \t{x = 1;}
    \not\to\ \ml{SET_VAR("x", CST 1)}
    \to\ \ml{SET(VAR "x", CST 1)}
    \item \t{t[0] = 1;}
    \not\to\ \ml{SET_ARRAY("t", CST 0, CST 1)}
    \to\ \ml{SET(OP2(S_INDEX, VAR "t", CST 0), CST 1)}
    \item \t{*x = 1}
    \not\to\ \ml{SET_DEREF("x", CST 1)}
    \to\ \ml{SET(OP1(M_DEREF, VAR "x"), CST 1)}
\end{itemize}

Justification:\\
With the addition of \t{*x}, \ml{SET_DEREF(x, e)} was first added but required much code duplication (all code related to assignment needed to appear thrice).\\
At first, \t{t[x][y] = 1;} was a parsing error, even though \t{t[x][y]} was a valid expression.\\
Since \t{M\_DEREF} added the horrible workaround \t{*(\&t[x][y]) = 1;}, allowing any expression to be assigned to, I decided it was time to allow more expressions to be treated as lvalues. Changing assignment was deemed the best course of action.\\
At the same time, former constructors \t{OPSET\_VAR}, \t{OPSET\_ARRAY} and \t{OPSET\_DEREF} were all merged into \t{OPSET}.

\subsection{Blocks}
This code
\begin{minted}{c}
{
    int x, y;
    x = 1;
}
\end{minted}
used to parse to
\begin{minted}{ocaml}
CBLOCK (
    [CDECL "x", CDECL "y"],
    [CEXPR (SET_VAR ("x", CST 1))]
)
\end{minted}
It now yields
\begin{minted}{ocaml}
CBLOCK [
    CLOCAL [("x", None); ("y", None)];
    CEXPR (SET (VAR "x", CST 1))
]
\end{minted}
Justification:\\
As soon as \t{int x;} and \t{int x = 1;} were allowed anywhere in the code and not just at the start of blocks, it made no more sense to have blocks carry information on the variables defined inside of them in a different form.

\subsection{Loops}
\begin{itemize}
    \item \t{for (e1; e2; e3) \{ c \}}
    \not\to\ \ml{e1; CWHILE (e2, c @ [e3])}
    \to\ \ml{e1; CWHILE (e2, c, e3, true)}
    \item \t{while (e) \{ c \}}
    \not\to\ \ml{CWHILE (e, c)}
    \to\ \ml{CWHILE (e, c, ESEQ [], true)}
    \item \t{do \{ c \} while (e);}
    \not\to\ (parsing error)
    \to\ \ml{CWHILE (e, c, ESEQ [], false)}
\end{itemize}
Justification:\\
The addition of do-while required some information on whether the test should be done at the start of the loop, a boolean was chosen.\\
At the same time, wanting to implement \t{break} and \t{continue}, I deemed it necessary to separate the body block and the finally clause of \t{for}. This is what the third argument accomplishes.

\subsection{Declarations}
New declaration typing:
\begin{minted}{ocaml}
type top_declaration =
  | CDECL of var_declaration * loc_expr option
  | CFUN of var_declaration * var_declaration list * loc_code
and var_declaration = Error.locator * string
and local_declaration = var_declaration * loc_expr option
\end{minted}

Justification:\\
The addition of optional initialisation values made no sense for function parameters, also \t{CDECL | CFUN} matches were required in many places where \t{CFUN} were impossible anyway. This resulted in too many ignored parameters or matches of the form \ml{| _ -> failwith "unreachable"} for my taste.\\
To streamline the typing \t{var\_declaration} was renamed to \t{top\_declaration} as an indication that it denotes a toplevel declaration. The name \t{var\_declaration} was reserved for function arguments, and \t{local\_declaration} for local variables.

\begin{minted}{c}
int x;               // CDECL ((_,"x"), None)
int y = 1;           // CDECL ((_,"y"), Some (CST 1))

int foo (int z) {    // CFUN (_, "foo", [(_,"z")], ...)
    int i;           // CLOCAL [((_,"i"), None)]
    int j = 0;       // CLOCAL [((_,"j"), Some (CST 0))]
}
\end{minted}
