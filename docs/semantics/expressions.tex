\section{Expressions}

\subsection{Reading values}
For local and global variables:
\[\dfrac{x\in\dom\rho \qquad \rho(x)\in\dom\mu}{\rho,\mu,\nil,v \gives^e \t{VAR}\ x\tto \mu,\nil,\mu^{64}(\rho(x))}\rule{Var}{}\]
i.e. reading a variable returns its contents and changes nothing to the memory.\\
\[\dfrac{\chi\ne\nil}{\rho,\mu,\chi,v \gives^e \t{VAR}\ x\tto \mu,\chi,v} \rule{Var}{^\chi}\]

For constant integers:
\[\dfrac{}{\rho,\mu,\nil,v \gives^e \t{CST}\ n\tto \mu,\nil,n}\rule{Cst}{}\]
\[\dfrac{\chi\ne\nil}{\rho,\mu,\chi,v \gives^e \t{CST}\ n\tto \mu,\chi,v} \rule{Cst}{^\chi}\]

For strings:
\[\dfrac{s\text{ stored at }a\in Addr}{\rho,\mu,\nil,v \gives^e \t{STRING}\ s\tto \mu,\nil,a}\rule{Str}{}\]
\[\dfrac{\chi\ne\nil}{\rho,\mu,\chi,v \gives^e \t{STRING}\ s\tto \mu,\chi,v} \rule{Str}{^\chi}\]

For arrays:
\[\dfrac{\begin{array}{c}
    \rho,\mu,\chi,v \gives^e i\tto \mu_i,\chi_i,v_i \\
    \rho,\mu_i,\chi_i,i \gives^e a\tto \mu_a,\nil,v_a \\
    v_a+v_i\times 8\in\dom\mu_a
\end{array}}{\rho,\mu,\chi,v \gives^e \t{OP2}(\t{S\_INDEX}, a, i) \tto \mu_a,\nil,\mu_a^{64}(v_a+v_i\times 8)}\rule{Idx}{}\]

None of these are different from the original \Cmm\ semantics.

\subsection{Unary operators without side-effects}
Unary minus (same as \Cmm):
\[\dfrac{
    \rho,\mu,\chi,v \gives^e e \tto \mu_e,\nil,v_e
}{\rho,\mu,\chi,v \gives^e \t{OP1}(\t{M\_MINUS}, e) \tto \mu_e,\nil,-v_e}\rule{Neg}{}\]

Unary bitwise negation (same as \Cmm):
\[\dfrac{
    \rho,\mu,\chi,v \gives^e e\tto \mu_e,\nil,v_e
}{\rho,\mu,\chi,v \gives^e \t{OP1}(\t{M\_NOT}, e) \tto \mu_e,\nil,-v_e-1}\rule{Not}{}\]


Indirection (added in \Cmp):
\[\dfrac{
    x\in\dom\rho
}{\rho,\mu,\nil,v \gives^e \t{OP1}(\t{M\_ADDR}, \t{VAR}\ x)\tto \mu,\nil,\rho(x)}\rule{Var}{^\&}\]

\[\dfrac{\begin{array}{c}
    \rho,\mu,\chi,v \gives^e i\tto \mu_i,\chi_i,v_i \\
    \rho,\mu_i,\chi_i,v_i \gives^e a\tto \mu_a,\nil,v_a \\
\end{array}}{\rho,\mu,\chi,v \gives^e \t{OP1}(\t{M\_ADDR}, \t{OP2}(\t{S\_INDEX}, a, i)) \tto \mu_a,\nil,v_a+v_i\times 8}\rule{Idx}{^\&}\]

\[\dfrac{
    \rho,\mu,\chi,v \gives^e a \tto \mu_a,\nil,v_a
}{\rho,\mu,\chi,v \gives^e \t{OP1}(\t{M\_ADDR}, \t{OP1}(\t{M\_DEREF}, a)) \tto \mu_a,\nil,v_a}\rule{Ptr}{^\&}\]

Dereferencing (added in \Cmp):
\[\dfrac{
    \rho,\mu,\chi,v \gives^e a\tto \mu_a,\nil,v_a \qquad v_a\in\dom\mu_a
}{\rho,\mu,\chi,v \gives^e \t{OP1}(\t{M\_DEREF}, a) \tto \mu_a,\nil,\mu_a^{64}(v_a)}\rule{Ptr}{}\]

When the operand raises a non-\nil\ flag:
\[\dfrac{
    \rho,\mu,\chi,v \gives^e e\tto \mu_e,\chi_e,v_e \qquad \chi_e\ne\nil
}{\rho,\mu,\chi,v \gives^e \t{OP1}(op, e) \tto \mu_e,\chi_e,v_e}\rule{Op1}{^\chi}\]

\subsection{Binary operators}
Multiplication (same as \Cmm):
\[\dfrac{\begin{array}{c}
    \rho,\mu,\chi,v \gives^e e_2\tto \mu_2,\chi_2,v_2 \\
    \rho,\mu_2,\chi_2,v_2 \gives^e e_1\tto \mu_1,\nil,v_1
\end{array}}{\rho,\mu,\chi,v \gives^e \t{OP2}(\t{S\_MUL}, e_1, e_2) \tto \mu_1,\nil,v_1\times v_2}\rule{Mul}{}\]

Addition (same as \Cmm):
\[\dfrac{\begin{array}{c}
    \rho,\mu,\chi,v \gives^e e_2\tto \mu_2,\chi_2,v_2 \\
    \rho,\mu_2,\chi_2,v_2 \gives^e e_1\tto \mu_1,\nil,v_1
\end{array}}{\rho,\mu,\chi,v \gives^e \t{OP2}(\t{S\_ADD}, e_1, e_2) \tto \mu_1,\nil,v_1 + v_2}\rule{Add}{}\]

Subtraction (same as \Cmm):
\[\dfrac{\begin{array}{c}
    \rho,\mu,\chi,v \gives^e e_2\tto \mu_2,\chi_2,v_2 \\
    \rho,\mu_2,\chi_2,v_2 \gives^e e_1\tto \mu_1,\nil,v_1
\end{array}}{\rho,\mu,\chi,v \gives^e \t{OP2}(\t{S\_SUB}, e_1, e_2) \tto \mu_1,\nil,v_1 - v_2}\rule{Sub}{}\]

Division and remainder (same as \Cmm):
\[\dfrac{\begin{array}{c}
    \rho,\mu,\chi,v \gives^e e_2\tto \mu_2,\chi_2,v_2 \qquad v_2\ne 0\\
    \rho,\mu_2,\chi_2,v_2 \gives^e e_1\tto \mu_1,\nil,v_1
\end{array}}{\rho,\mu,\chi,v \gives^e \t{OP2}(\t{S\_DIV}, e_1, e_2) \tto \mu_1,\nil,v_1\div v_2}\rule{Div}{}\]
\[\dfrac{\begin{array}{c}
    \rho,\mu,\chi,v \gives^e e_2\tto \mu_2,\chi_2,v_2 \qquad v_2\ne 0\\
    \rho,\mu_2,\chi_2,v_2 \gives^e e_1\tto \mu_1,\nil,v_1
\end{array}}{\rho,\mu,\chi,v \gives^e \t{OP2}(\t{S\_MOD}, e_1, e_2) \tto \mu_1,\nil,v_1\mod v_2}\rule{Mod}{}\]

Shifts (added in \Cmp):
\[\dfrac{\begin{array}{c}
    \rho,\mu,\chi,v \gives^e e_2\tto \mu_2,\chi_2,v_2 \\
    \rho,\mu_2,\chi_2,v_2 \gives^e e_1\tto \mu_1,\nil,v_1
\end{array}}{\rho,\mu,\chi,v \gives^e \t{OP2}(\t{S\_SHL}, e_1, e_2) \tto \mu_1,\nil,v_1\times 2^{v_2}}\rule{Shl}{}\]
\[\dfrac{\begin{array}{c}
    \rho,\mu,\chi,v \gives^e e_2\tto \mu_2,\chi_2,v_2 \\
    \rho,\mu_2,\chi_2,v_2 \gives^e e_1\tto \mu_1,\nil,v_1
\end{array}}{\rho,\mu,\chi,v \gives^e \t{OP2}(\t{S\_SHR}, e_1, e_2) \tto \mu_1,\nil,v_1\div 2^{v_2}} \rule{Shr}{}\]

Let \(\dec: \{ \bot, \top \}^{64} \to \ZZ_{64}\) the function
\[(b_0, \cdots, b_{63}) \mapsto \sum_{i=0}^{63} (1\t{ if } b_i\t{ else }0)\times 2^i\]
and \(\bin = \dec^{-1}\).

We can now define bitwise operators as follows (added in \Cmp).\\
\[\dfrac{\begin{array}{c}
    \rho,\mu,\chi,v \gives^e e_2\tto \mu_2,\chi_2,v_2 \qquad \rho,\mu_2,\chi_2,v_2\gives^e e_1\tto \mu_1,\nil,v_1 \\
    (b_0^2,\cdots,b_{63}^2) = \bin(v_2) \qquad (b_0^1,\cdots,b_{63}^1) = \bin(v_1)
\end{array}}{\rho,\mu,\chi,v \gives^e \t{OP2}(\t{S\_AND}, e_1, e_2) \tto \mu_1,\nil,\dec(b_0^1 \wedge b_0^2,\cdots, b_{63}^1 \wedge b_{63}^2),\mu_1} \rule{And}{}\]
\[\dfrac{\begin{array}{c}
    \rho,\mu,\chi,v \gives^e e_2\tto \mu_2,\chi_2,v_2 \qquad \rho,\mu_2,\chi_2,v_2\gives^e e_1\tto \mu_1,\nil,v_1 \\
    (b_0^2,\cdots,b_{63}^2) = \bin(v_2) \qquad (b_0^1,\cdots,b_{63}^1) = \bin(v_1)
\end{array}}{\rho,\mu,\chi,v \gives^e \t{OP2}(\t{S\_OR}, e_1, e_2) \tto \mu_1,\nil,\dec(b_0^1 \vee b_0^2,\cdots, b_{63}^1 \vee b_{63}^2),\mu_1} \rule{Ior}{}\]
\[\dfrac{\begin{array}{c}
    \rho,\mu,\chi,v \gives^e e_2\tto \mu_2,\chi_2,v_2 \qquad \rho,\mu_2,\chi_2,v_2\gives^e e_1\tto \mu_1,\nil,v_1 \\
    (b_0^2,\cdots,b_{63}^2) = \bin(v_2) \qquad (b_0^1,\cdots,b_{63}^1) = \bin(v_1)
\end{array}}{\rho,\mu,\chi,v \gives^e \t{OP2}(\t{S\_XOR}, e_1, e_2) \tto \mu_1,\nil,\dec(b_0^1 \oplus b_0^2,\cdots, b_{63}^1 \oplus b_{63}^2),\mu_1} \rule{Xor}{}\]

When one of the operands raises a non-\nil\ flag:
\[\dfrac{\begin{array}{c}
    \rho,\mu,\chi,v \gives^e e_2\tto \mu_2,\chi_2,v_2 \\
    \rho,\mu_2,\chi_2,v_2 \gives^e e_1\tto \mu_1,\chi_1,v_1 \\
    \chi_1\ne\nil
\end{array}}{\rho,\mu,\chi,v \gives^e \t{OP2}(op, e_1, e_2) \tto \mu_1,\chi_1,v_1} \rule{Op2}{^\chi}\]

\subsection{Comparisons}
All are the same as in \Cmm.
\[\dfrac{\begin{array}{c}
    \rho,\mu,\chi,v \gives^e e_2\tto \mu_2,\chi_2,v_2 \\
    \rho,\mu_2,\chi_2,v \gives^e e_1\tto \mu_1,\nil,v_1 \\
    \qquad v_1 = v_2
\end{array}}{\rho,\mu,\nil,v \gives^e \t{CMP}(\t{C\_EQ}, e_1, e_2) \tto \mu_1,\nil,1}\rule{Eq}{^\top}\]
\[\dfrac{\begin{array}{c}
    \rho,\mu,\chi,v \gives^e e_2\tto \mu_2,\chi_2,v_2 \\
    \rho,\mu_2,\chi_2,v \gives^e e_1\tto \mu_1,\nil,v_1 \\
    \qquad v_1 < v_2
\end{array}}{\rho,\mu,\nil,v \gives^e \t{CMP}(\t{C\_LT}, e_1, e_2) \tto \mu_1,\nil,1}\rule{Lt}{^\top}\]
\[\dfrac{\begin{array}{c}
    \rho,\mu,\chi,v \gives^e e_2\tto \mu_2,\chi_2,v_2 \\
    \rho,\mu_2,\chi_2,v \gives^e e_1\tto \mu_1,\nil,v_1 \\
    \qquad v_1 \leq v_2
\end{array}}{\rho,\mu,\nil,v \gives^e \t{CMP}(\t{C\_LE}, e_1, e_2) \tto \mu_1,\nil,1}\rule{Le}{^\top}\]
\[\dfrac{\begin{array}{c}
    \rho,\mu,\chi,v \gives^e e_2\tto \mu_2,\chi_2,v_2 \\
    \rho,\mu_2,\chi_2,v \gives^e e_1\tto \mu_1,\nil,v_1 \\
    \qquad v_1 \ne v_2
\end{array}}{\rho,\mu,\nil,v \gives^e \t{CMP}(\t{C\_EQ}, e_1, e_2) \tto \mu_1,\nil,0}\rule{Eq}{^\bot}\]
\[\dfrac{\begin{array}{c}
    \rho,\mu,\chi,v \gives^e e_2\tto \mu_2,\chi_2,v_2 \\
    \rho,\mu_2,\chi_2,v \gives^e e_1\tto \mu_1,\nil,v_1 \\
    \qquad v_1 \not< v_2
\end{array}}{\rho,\mu,\nil,v \gives^e \t{CMP}(\t{C\_LT}, e_1, e_2) \tto \mu_1,\nil,0}\rule{Lt}{^\bot}\]
\[\dfrac{\begin{array}{c}
    \rho,\mu,\chi,v \gives^e e_2\tto \mu_2,\chi_2,v_2 \\
    \rho,\mu_2,\chi_2,v \gives^e e_1\tto \mu_1,\nil,v_1 \\
    \qquad v_1 \not\leq v_2
\end{array}}{\rho,\mu,\nil,v \gives^e \t{CMP}(\t{C\_LE}, e_1, e_2) \tto \mu_1,\nil,0}\rule{Le}{^\bot}\]

For optimisation purposes mostly, the comparison operators \(\t{C\_NE}\), \(\t{C\_GT}\), \(\t{C\_GE}\) may be introduced by the compiler (not by the parser, however).\\
They are defined as
\[\dfrac{\begin{array}{c}
    \rho,\mu,\chi,v \gives^e \t{OP1}(\t{M\_NOT}, \t{CMP}(\t{C\_EQ}, e_1, e_2)) \tto \mu',\chi',v'
\end{array}}{\rho,\mu,\nil,v \gives^e \t{CMP}(\t{C\_NE}, e_1, e_2) \tto \mu',\chi',v'}\rule{Ne}{}\]
\[\dfrac{\begin{array}{c}
    \rho,\mu,\chi,v \gives^e \t{OP1}(\t{M\_NOT}, \t{CMP}(\t{C\_LE}, e_1, e_2)) \tto \mu',\chi',v'
\end{array}}{\rho,\mu,\nil,v \gives^e \t{CMP}(\t{C\_GT}, e_1, e_2) \tto \mu',\chi',v'}\rule{Gt}{}\]
\[\dfrac{\begin{array}{c}
    \rho,\mu,\chi,v \gives^e \t{OP1}(\t{M\_NOT}, \t{CMP}(\t{C\_LT}, e_1, e_2)) \tto \mu',\chi',v'
\end{array}}{\rho,\mu,\nil,v \gives^e \t{CMP}(\t{C\_GE}, e_1, e_2) \tto \mu',\chi',v'}\rule{Ge}{}\]

When one of the operands raises a non-\nil\ flag:
\[\dfrac{\begin{array}{c}
    \rho,\mu,\chi,v \gives^e e_2\tto \mu_2,\chi_2,v_2 \\
    \rho,\mu_2,\chi_2,v_2 \gives^e e_1\tto \mu_1,\chi_1,v_1 \\
    \chi_1\ne\nil
\end{array}}{\rho,\mu,\chi,v \gives^e \t{CMP}(op, e_1, e_2) \tto \mu_1,\chi_1,v_1} \rule{Cmp}{^\chi}\]

\subsection{Assignments}

\[\dfrac{\begin{array}{c}
    \rho,\mu,\chi,v \gives^e e \tto \mu_e,\chi_e,v_e \\
    \rho,\mu_e,\chi_e,v_e \gives^e \t{OP1}(\t{M\_ADDR}, a) \tto \mu_a,\nil,v_a
\end{array}}{\rho,\mu,\chi,v \gives^e \t{SET}(a, e) \tto \mu_a^{64}[v_a\mapsto v_e],\nil,v_e}\rule{Ptr}{^\gets}\]
\[\dfrac{\begin{array}{c}
    \rho,\mu,\chi,v \gives^e e \tto \mu_e,\chi_e,v_e \\
    \rho,\mu_e,\chi_e,v_e \gives^e \t{OP1}(\t{M\_ADDR}, a) \tto \mu_a,\chi_a,v_a \qquad \chi_a\ne\nil
\end{array}}{\rho,\mu,\chi,v \gives^e \t{SET}(a, e) \tto \mu_a,\chi_a,v_a}\rule{Ptr}{^{\gets\chi}}\]

\subsection{Increments}
\[\dfrac{\begin{array}{c}
    \rho,\mu,\chi,v \gives^e \t{OP1}(\t{M\_ADDR}, a)\tto \mu_a,\nil,v_a \\
    k = \mu_a^{64}(v_a)
\end{array}}{\rho,\mu,\chi,v \gives^e \t{OP1}(\t{M\_POST\_INC}, a) \tto \mu_a^{64}[v_a\mapsto k+1],\nil,k} \rule{Post}{^\uparrow}\]
\[\dfrac{\begin{array}{c}
    \rho,\mu,\chi,v \gives^e \t{OP1}(\t{M\_ADDR}, a)\tto \mu_a,\nil,v_a \\
    k = \mu_a^{64}(v_a)
\end{array}}{\rho,\mu,\chi,v \gives^e \t{OP1}(\t{M\_POST\_DEC}, a) \tto \mu_a^{64}[v_a\mapsto k-1],\nil,k} \rule{Post}{^\downarrow}\]
\[\dfrac{\begin{array}{c}
    \rho,\mu,\chi,v \gives^e \t{OP1}(\t{M\_ADDR}, a)\tto \mu_a,\nil,v_a \\
    k = \mu_a^{64}(v_a)
\end{array}}{\rho,\mu,\chi,v \gives^e \t{OP1}(\t{M\_PRE\_INC}, a) \tto \mu_a^{64}[v_a\mapsto k+1],\nil,k+1} \rule{Pre}{^\uparrow}\]
\[\dfrac{\begin{array}{c}
    \rho,\mu,\chi,v \gives^e \t{OP1}(\t{M\_ADDR}, a)\tto \mu_a,\nil,v_a \\
    k = \mu_a^{64}(v_a)
\end{array}}{\rho,\mu,\chi,v \gives^e \t{OP1}(\t{M\_PRE\_DEC}, a) \tto \mu_a^{64}[v_a\mapsto k-1],\nil,k-1} \rule{Pre}{^\downarrow}\]

\subsection{Extended assignments}
Let \(op\in \t{bin\_op}\setminus \{ \t{S\_INDEX} \}\).

\[\dfrac{\begin{array}{c}
    \rho,\mu,\chi,v \gives^e e \tto \mu_e,\chi_e,v_e \\
    \rho,\mu_e,\chi_e,v_e \gives^e \t{OP1}(\t{M\_ADDR}, a) \tto \mu_a,\chi_a,v_a \\
    v_a\in\dom\mu_a \qquad \mu_a^{64}(v_a) = u \\
    \rho,\mu_a,\chi_a,v_a \gives^e \t{OP2}(op, \t{CST}\ v_a, \t{CST}\ v_e) \tto \mu',\nil,w
\end{array}}{\rho,\mu,\chi,v \gives^e \t{OPSET}(op, a, e) \tto \mu'[v_a\mapsto w],\nil,w} \rule{Opset}{}\]
\[\dfrac{\begin{array}{c}
    \rho,\mu,\chi,v \gives^e e\tto \mu_e,\chi_e,v_e \\
    \rho,\mu_e,\chi_e,v_e \gives^e \t{OP1}(\t{M\_ADDR}, a) \tto \mu_a,\chi_a,v_a \\
    \chi_a\ne\nil \\
\end{array}}{\rho,\mu,\chi,v \gives^e \t{OPSET}(op, a, e) \tto \mu_a,\chi_a,v_a} \rule{Opset}{^\chi}\]

\subsection{Ternary operator}
\[\dfrac{\begin{array}{c}
    \rho,\mu,\chi,v \gives^e e\tto \mu_e,\nil,v_e \qquad v_e = 0\\
    \rho,\mu_e,\nil,v_e \gives^e e_\bot \tto \mu_\bot,\chi_\bot,v_\bot
\end{array}}{\rho,\mu,\chi,v \gives^e \t{EIF}(e, e_\top, e_\bot) \tto \mu_\bot,\chi_\bot,v_\bot}\rule{Tern}{^\bot}\]

\[\dfrac{\begin{array}{c}
    \rho,\mu,\chi,v \gives^e e\tto \mu_e,\nil,v_e \qquad v_e \ne 0\\
    \rho,\mu_e,\nil,v_e \gives^e e_\top \tto \mu_\top,\chi_\top,v_\top \\
\end{array}}{\rho,\mu,\chi,v \gives^e \t{EIF}(e, e_\top, e_\bot) \tto \mu_\top,\chi_\top,v_\top}\rule{Tern}{^\top}\]

\[\dfrac{\begin{array}{c}
    \rho,\mu,\chi,v \gives^e e\tto \mu_e,\chi_e,v_e \qquad \chi_e\ne\nil
\end{array}}{\rho,\mu,\chi,v \gives^e \t{EIF}(e, e_\top, e_\bot) \tto \mu_e,\chi_e,v_e}\rule{Tern}{^\chi}\]


\subsection{Sequence}
\[\dfrac{\begin{array}{c}
    \rho,\mu_0,\chi_0,v_0 \gives^e e_1 \tto \mu_1,\chi_1,v_1 \\
    \cdots \\
    \rho,\mu_{n-1},\chi_{n-1},v_{n-1} \gives^e e_n \tto \mu_n,\chi_n,v_n \\
\end{array}}{\rho,\mu_0,\chi_0,v_0 \gives^e \t{ESEQ}\ [e_1;\cdots;e_n] \tto \mu_n,\chi_n,v_n} \rule{Seq}{^n}\]

\subsection{Function call}
Works for both a toplevel function and a function pointer:
\[\dfrac{\begin{array}{c}
    \rho,\mu_{n+1},\chi_{n+1},v_{n+1} \gives^e e_n \tto \mu_n,\chi_n,v_n \\
    \cdots \\
    \rho,\mu_2,\chi_2,v_2 \gives^e e_1 \tto \mu_1,\nil,v_1 \\
    f\in\dom\rho \qquad \rho(f)\in\dom\fun \\
    \rho_g,\mu_1,\nil,0 \gives^e \fun(\rho(f))(v_1, \cdots, v_n) \tto \rho_f,\mu_f,\chi_f,v_f
\end{array}}{\rho,\mu_n,\chi_n,v_n \gives^e \t{CALL}(f, [e_1; \cdots; e_n]) \tto \mu_f,\chi_f,v_f} \rule{Call}{^n}\]

\[\dfrac{\begin{array}{c}
    \rho,\mu_{n+1},\chi_{n+1},v_{n+1} \gives^e e_n \tto \mu_n,\chi_n,v_n \\
    \cdots \\
    \rho,\mu_2,\chi_2,v_2 \gives^e e_1 \tto \mu_1,\chi_1,v_1 \\
    \chi_1\ne\nil \\
\end{array}}{\rho,\mu_{n+1},\chi_{n+1},v_{n+1} \gives^e \t{CALL}(f, [e_1; \cdots; e_n]) \tto \mu_1,\chi_1,v_1} \rule{Call}{^\chi}\]
