\section{Major design choices}

\subsection{Compiler passes}

The compiler does the following passes:
\begin{enumerate}
    \item lex and parse source code into ASM
    \item reduce expressions
    \item generate abstract assembler
    \item translate to text
    \item print aligned instructions
    \item assemble and link
\end{enumerate}

Reduction and generation are partly merged : each time an expression is encountered it is reduced (only once) before being generated.

The parsing step produces a list of declarations, which the reduction slightly modifies. The generation outputs the abstract assembler defined in \t{generate.ml}. Translation and printing combined produce ascii text. The assembly step is external.

\subsection{File distribution}

\begin{itemize}
    \item \t{cAST.ml}: text to AST translation
    \item \t{compile.ml}: AST to \t{program}
    \item \t{generate.ml}: \t{program} to \t{alignment} to string representation
    \item \t{reduce.ml}: expression reduction
    \item \t{pigment.ml}: color management
\end{itemize}

\subsection{Scope management}

The scope is managed throughout \t{compile.ml} with the three parameters
\begin{itemize}
    \item \t{envt}, an associative list that yields a location for each identifier
    \item \t{depth}, an integer that tells the current number of local or temporary variables on the stack
    \item \t{va\_depth}, either \t{None} or the number of non-optional parameters
\end{itemize}

To locate a variable, \t{envt} is scanned and yields a \t{location}:
\begin{minted}{ocaml}
type location =
    | Stack of int (* variable is on the stack at the given depth *)
    | Const of int (* known constant value *)
    | Globl of string (* global variable *)
    | FnPtr of string (* function pointer *)
    | Hexdc of string (* hexadecimal constant *)
    | Regst of register (* variable is in a register *)
    | Deref of register (* variable is at the address given by a register *)
    | Index of register * register (* variable is at an index with offset given by the two registers *)
\end{minted}

\subsection{Register distribution}

Throughout the main function of \t{compile.ml}, the following conventions are used:
\begin{itemize}
    \item \t{RAX} contains the last evaluated expression
    \item \t{RDI} contains the last evaluated address
    \item \t{RCX} is an extra register for 2-register operations
    \item \t{R10} is used whenever a function pointer is needed
\end{itemize}

\subsection{Tag management}
Tags also have rules:
\begin{itemize}
    \item \t{.LC}\(n\) are strings
    \item \t{.EX}\(n\) are exception names (except for \(n\in \{ 0, 1 \}\) reserved for exception formatting strings)
    \item \t{.eaddr} is the address of the current exception handler
    \item \t{.ebase} is the base pointer of that handler
    \item \(f\t{.}i\t{\_}l\) is the usual format for the \(i\)'th tag in function \(f\) of type \(l\). Values for \(l\) include \t{return}, \t{loop\_start}, \t{switch\_done}, ...
\end{itemize}

Some parameters are passed around in the generating functions to record which tags to jump to at various steps.
