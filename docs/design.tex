\section{Major design choices}

\subsection{Compiler passes}

The compiler does the following passes:
\begin{enumerate}
    \item lex and parse source code into ASM
    \item reduce expressions
    \item generate abstract assembler
    \item translate to text
    \item print aligned instructions
    \item assemble and link
\end{enumerate}

Reduction and generation are partly merged : each time an expression is encountered it is reduced (only once) before being generated.

The parsing step produces a list of declarations, which the reduction slightly modifies. The generation outputs the abstract assembler defined in \t{generate.ml}. Translation and printing combined produce ascii text. The assembly step is external.

\subsection{File distribution}

Roles are distributed as such:
\begin{itemize}
    \item \t{cAST.ml}: text to AST translation
    \item \t{compile.ml}: AST to \t{program}
    \item \t{generate.ml}: \t{program} to \t{alignment} to string representation
    \item \t{reduce.ml}: expression reduction
    \item \t{pigment.ml}: color management
\end{itemize}

\subsection{Scope management}

The scope is managed throughout \t{compile.ml} with the three parameters
\begin{itemize}
    \item \t{envt}, an associative list that yields a location for each identifier
    \item \t{depth}, an integer that tells the current number of local or temporary variables on the stack
    \item \t{va\_depth}, either \t{None} or the number of non-optional parameters
\end{itemize}

To locate a variable, \t{envt} is scanned and yields a \t{location}, which denotes any of the supported ways to access the value/address of a variable.

\subsection{Register distribution}

Throughout the main function of \t{compile.ml}, the following conventions are used:
\begin{itemize}
    \item \t{RAX} contains the last evaluated expression
    \item \t{RDI} contains the last evaluated address
    \item \t{RCX} is an extra register for 2-register operations
    \item \t{R10} is used whenever a function pointer is needed
\end{itemize}

\subsection{Tag management}
Tags also have rules:
\begin{itemize}
    \item \t{.LC}\(n\) are strings
    \item \t{.EX}\(n\) are exception names (except for \(n\in \{ 0, 1 \}\) reserved for exception formatting strings)
    \item \t{.eaddr} is the address of the current exception handler
    \item \t{.ebase} is the base pointer of that handler
    \item \(f\t{.}i\t{\_}l\) is the usual format for the \(i\)'th tag in function \(f\) of type \(l\). Values for \(l\) include \t{return}, \t{loop\_start}, \t{switch\_done}, ...
\end{itemize}

Some parameters are passed around in the generating functions to record which tags to jump to at various steps.

\subsection{Function skeleton}
All functions have the same basic structure, they :
\begin{itemize}
    \item start by storing their callee-saved registers
    \item keep their frame pointer equal to their base pointer except during subroutine calls
    \item have a \t{return} tag that they jump to in order to return
\end{itemize}

\subsection{For convenience}
To avoid some mistakes and avoid redefining all common constant values, the compiler includes a fixed list of known functions selected from the standard library and some constant values. These functions are spell-checked, and their number of arguments is validated.\\
Constants include common values such as \t{NULL} and \t{EOF} as well as non-standard ones like \t{BYTE} (255).

\subsection{Exception management}
Exceptions are handled roughly as follows.\\
When a \t{try} block is encountered a handler is built: the current value for \t{\%rbp} is stored in \t{.ebase}, the tag to jump to in the case of a \t{throw} is stored in \t{.eaddr}. A similar handler is set up at the start of main.\\
If the block exits without an exception, the handler is restored to its previous value (\t{.eaddr} and \t{.ebase} were saved on the stack during setup) and the \t{finally} clause is executed.\\
If an exception is raised the \t{throw} instruction restores the base pointer and jumps to the handler address. The handler is saved in temporary registers then restored to its previous value. Handler name (determined by the address at which it is stored) is compared sequentially to the handlers. A variable binding is created with the value thrown. There are two exceptions to these rules: \t{\_} as an exception name can match any exception (but can't be thrown); \t{\_} as a variable produces no binding. A \t{catch (\_) \{\}} statement is a surefire way to ensure the program will not crash.\\
The emergency handler (setup during main) flushes the output then prints its own error message with the name and value of the exception. It uses a criteria on the value to guess whether it should be printed as an integer or a string. Strings created dynamically will wrongly be displayed as integers, and integers that happen to equal an address at which a static string is stored will wrongly be displayed as strings.
