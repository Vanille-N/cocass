\section{Modifications}
The following changes were made to existing constructs of the AST.\\
Each comes with an explanation of why it was deemed necessary in the context of other additions to the language.

\subsection{Assignments}
\t{SET} replaces both \t{SET\_VAR} and \t{SET\_ARRAY} as well as what was for some time \t{SET\_DEREF}.\\
\begin{itemize}
    \item \t{x = 1;}
    \not\to\ \ml{SET_VAR("x", CST 1)}
    \to\ \ml{SET(VAR "x", CST 1)}
    \item \t{t[0] = 1;}
    \not\to\ \ml{SET_ARRAY("t", CST 0, CST 1)}
    \to\ \ml{SET(OP2(S_INDEX, VAR "t", CST 0), CST 1)}
    \item \t{*x = 1}
    \not\to\ \ml{SET_DEREF("x", CST 1)}
    \to\ \ml{SET(OP1(M_DEREF, VAR "x"), CST 1)}
\end{itemize}

Justification:\\
With the addition of \t{*x}, \ml{SET_DEREF(x, e)} was first added but required much code duplication (all code related to assignment needed to appear thrice).\\
At first, \t{t[x][y] = 1;} was a parsing error, even though \t{t[x][y]} was a valid expression.\\
Since \t{M\_DEREF} added the horrible workaround \t{*(\&t[x][y]) = 1;}, allowing any expression to be assigned to, I decided it was time to allow more expressions to be treated as lvalues. Changing assignment was deemed the best course of action.\\
At the same time, former constructors \t{OPSET\_VAR}, \t{OPSET\_ARRAY} and \t{OPSET\_DEREF} were all merged into \t{OPSET}.

\end{document}
