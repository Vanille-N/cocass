\documentclass[10pt,a4paper]{article}
\usepackage[utf8]{inputenc}
\usepackage[T1]{fontenc}
\usepackage{geometry}
\usepackage[french]{babel}
\usepackage{amsmath}
\usepackage{amssymb}
\usepackage{tcolorbox}
\usepackage{amsfonts}
\usepackage{stmaryrd}
\usepackage{xcolor}
\usepackage{algorithm}
\usepackage{algpseudocode}
\usepackage{mathtools}
\usepackage{dsfont}
\usepackage{tikz}
\usepackage{tikz-cd}
\usepackage{hyperref}


\geometry{top=10mm, left=15mm, right=15mm, bottom=20mm}

\newcommand{\dd}{\text{d}}
\renewcommand{\vec}{\overrightarrow}
\renewcommand{\Im}{\mathfrak{Im}\ }
\newcommand{\Ker}{\text{Ker}\ }
\newcommand{\bijective}{%
  \hookrightarrow\mathrel{\mspace{-15mu}}\rightarrow
}
\newcommand{\surjective}{\twoheadrightarrow}
\newcommand{\injective}{\hookrightarrow}
\newcommand{\implication}{\Longrightarrow}
\newcommand{\reciprocal}{\Longleftarrow}
\newcommand{\equivalent}{\Longleftrightarrow}
\newcommand{\NN}{\ensuremath{\mathbb{N}}}
\newcommand{\RR}{\ensuremath{\mathbb{R}}}
\newcommand{\QQ}{\ensuremath{\mathbb{Q}}}
\newcommand{\ZZ}{\ensuremath{\mathbb{Z}}}
\newcommand{\CC}{\ensuremath{\mathbb{C}}}
\renewcommand{\epsilon}{\varepsilon}
\newcommand{\floor}[1]{\left\lfloor #1 \right\rfloor}
\newcommand{\ceil}[1]{\left\lceil #1 \right\rceil}
\newcommand{\comb}[2]{\displaystyle{#2 \choose #1}}
\newcommand{\parts}{\mathcal{P}}
\newcommand{\permut}{\mathfrak{S}}
\newcommand{\id}{\text{Id}}
\newcommand{\indic}{\mathds{1}}
\renewcommand{\leq}{\leqslant}
\renewcommand{\geq}{\geqslant}
\newcommand{\mat}[2]{\mathcal{M}_{#1}(#2)}

\DeclareMathOperator{\mut}{mut}
\DeclareMathOperator{\stab}{Stab}
\renewcommand{\time}{\textsc{Time}}
\newcommand{\len}{\textsc{Length}}
\newcommand{\call}[1]{\textsc{#1}}
\newcommand{\bbrack}[1]{\left\llbracket#1\right\rrbracket}

\makeatletter
\renewcommand{\Function}[2]{%
  \csname ALG@cmd@\ALG@L @Function\endcsname{#1}{#2}%
  \def\jayden@currentfunction{#1}%
}
\newcommand{\funclabel}[1]{%
  \@bsphack
  \protected@write\@auxout{}{%
    \string\newlabel{#1}{{\jayden@currentfunction}{\thepage}}%
  }%
  \@esphack
}
\makeatother

\definecolor{green}{rgb}{0.0, 0.5, 0.0}
\makeatletter\algnewcommand{\Assert}[1]{\State {\color{green} \qquad$\{ #1 \}$}}\makeatother
\makeatletter\algnewcommand{\Therefore}[1]{\State {\color{green} \qquad$\therefore\{ #1 \}$}}\makeatother
\makeatletter\algnewcommand{\Also}[1]{{\color{green}$\wedge\{ #1 \}$}}\makeatother
\makeatletter\algnewcommand{\AAssert}[1]{\State {\color{red} \qquad$\{ #1 \}$}}\makeatother
\makeatletter\algnewcommand{\AAlso}[1]{{\color{red}$\wedge\{ #1 \}$}}\makeatother


\newcommand{\opnorm}[1]{%
  \left|\mkern-1.5mu\left|\mkern-1.5mu\left|
   #1
  \right|\mkern-1.5mu\right|\mkern-1.5mu\right|
}
\newcommand\sbullet[1][.5]{\mathbin{\vcenter{\hbox{\scalebox{#1}{$\bullet$}}}}}

\newlength\tindent
\setlength{\tindent}{\parindent}
\setlength{\parindent}{0pt}
\renewcommand{\indent}{\hspace*{\tindent}}
\setlength\parindent{0pt}

\DeclareMathOperator{\dom}{dom}
\newcommand{\struct}[1]{\texttt{#1}}
\renewcommand{\mod}{\text{ mod }}
\newcommand{\tto}{\Rightarrow}


\hypersetup{
    colorlinks=true,
    linkcolor=blue,
    filecolor=magenta,
    urlcolor=blue,
}
\urlstyle{tt}


\newcommand{\Cmp}{\(\pm\)C)}
\newcommand{\Cmm}{C\textminus\textminus}
\newcommand{\fun}{\text{fun}_\pi^n}
\newcommand{\fnptr}{\text{fun}_\mu^n}
\newcommand{\gives}{\vdash_\pi}

\title{Semantics of \Cmp (i.e. extended \Cmm)}
\author{\textsc{Villani} Neven, ENS Paris-Saclay\\\textsc{Programmation 1 -- Compilateur COCass}}

\begin{document}
\maketitle

Retranscribed and extended from \url{http://www.lsv.fr/~goubault/CoursProgrammation/minic.html}\\
and \url{http://www.lsv.fr/~goubault/CoursProgrammation/prog1_sem1.pdf}.\\


\section{Expressions}

\subsection*{Notation}

\(\ZZ_{64}\triangleq \ZZ/64\ZZ\) is the set in which all calculations are done.\\

We write \((\rho: \mathcal{S}\to\ZZ_{64})\in \mathcal{P}\) the environment, where \(\mathcal{S}\) is the set of names of variables and functions, \((\mu: \ZZ_{64}\to\ZZ_{64})\in \mathcal{M}\) the memory.\\

A flag is defined as an element of \(\mathcal{E} \triangleq S \sqcup \{ \t{brk}, \t{ret}, \t{cnt}, \t{nil} \}\).\\
Intuitively, \(\rho,\mu,\chi,v \Vdash_\pi c \tto \rho',\mu',\chi',v'\) means that when \(c\) is executed under the environment \(\rho\) with the memory \(\mu\), the flag \(\chi\), and the previous value \(v\), it updates it to the new environment and memory \(\rho'\) and \(\mu'\), raises \(\chi'\), and changes the value to \(v'\).\\

In addition, we write \(\fun: \mathcal{S}\to \mathcal{F}^n\) where \(\mathcal{F}^n \triangleq (\mathcal{M} \times \mathcal{E} \times \ZZ_{64})^{\mathcal{P} \times \mathcal{M} \times \ZZ_{64}^n}\), i.e. functions that take an environment, a memory layout and \(n\) 64-bit integer arguments and return one 64-bit integer, the updated memory, and a flag.\\
\(\fnptr: \ZZ_{64}\to \mathcal{F}^n\) returns the function (if there is one) defined at the given memory address, and is useful for variables that are function pointers.\\

For \(\mu\in\mathcal{M}, v\in\ZZ_{64}, x\in\ZZ_{64}\) we write \(\mu[x\mapsto v]: \left\{\begin{array}{ll}x\mapsto v & \\ y\mapsto \mu(y) & y\in \dom\mu\setminus\{x\}\end{array}\right.\)

\subsection{Reading values}
For local and global variables:
\[\dfrac{x\in\dom\rho \qquad \rho(x)\in\dom\mu}{\rho,\mu,\t{nil},v \gives \t{VAR}\ x\tto \rho,\mu,\t{nil},\mu(\rho(x))}\rule{Var}{}\]
i.e. reading a variable returns its contents and changes nothing to the memory.\\
\[\dfrac{\chi\ne\t{nil}}{\rho,\mu,\chi,v \gives \t{VAR}\ x\tto \rho,\mu,\chi,v} \rule{Var}{^\chi}\]

For constant integers:
\[\dfrac{}{\rho,\mu,\t{nil},v \gives \t{CST}\ n\tto \rho,\mu,\t{nil},n}\rule{Cst}{}\]
\[\dfrac{\chi\ne\t{nil}}{\rho,\mu,\chi,v \gives \t{CST}\ n\tto \rho,\mu,\chi,v} \rule{Cst}{^\chi}\]

For strings:
\[\dfrac{s\text{ stored at }a\in Addr}{\rho,\mu,\t{nil},v \gives \t{STRING}\ s\tto \rho,\mu,\t{nil},a}\rule{Str}{}\]
\[\dfrac{\chi\ne\t{nil}}{\rho,\mu,\chi,v \gives \t{STRING}\ s\tto \rho,\mu,\chi,v} \rule{Cst}{^\chi}\]

For arrays:
\[\dfrac{\begin{array}{c}
    \rho,\mu,\chi,v \gives i\tto \rho,\mu_i,\chi_i,v_i \\
    \rho,\mu_i,\chi_i,i \gives a\tto \rho,\mu',\t{nil},v_a \\
    v_a+v_i\times 8\in\dom\mu'
\end{array}}{\rho,\mu,\t{nil},v \gives \t{OP2}(\t{S\_INDEX}, a, i) \tto \rho,\mu',\t{nil},\mu'(v_a+v_i\times 8)}\rule{Idx}{}\]

None of these are different from the original \Cmm semantics.

\subsection{Unary operators without side-effects}
Unary minus (same as \Cmm):
\[\dfrac{
    \rho,\mu,\chi,v \gives e \tto \rho,\mu',\t{nil},v_e
}{\rho,\mu,\chi,v \gives \t{OP1}(\t{M\_MINUS}, e) \tto \rho,\mu',\t{nil},-v_e}\rule{Neg}{}\]

Unary bitwise negation (same as \Cmm):
\[\dfrac{
    \rho,\mu,\chi,v \gives e\tto \rho,\mu',\t{nil},v_e
}{\rho,\mu,\chi,v \gives \t{OP1}(\t{M\_NOT}, e) \tto \rho,\mu',t{nil},-v_e-1}\rule{Not}{}\]


Indirection (added in \Cmp):
\[\dfrac{
    x\in\dom\rho
}{\rho,\mu,\t{nil},v \gives \t{OP1}(\t{M\_ADDR}, x)\tto \rho,\mu,\t{nil},\rho(x)}\rule{Var}{^\&}\]

\[\dfrac{\begin{array}{c}
    \rho,\mu,\chi,v \gives i\tto \rho,\mu_i,\chi_i,v_i \\
    \rho,\mu_i,\chi_i,v_i \gives a\tto \rho,\mu',\t{nil},v_a
\end{array}}{\rho,\mu,\chi,v \gives \t{OP1}(\t{M\_ADDR}, \t{OP2}(\t{S\_INDEX}, a, e)) \tto \rho,\mu',\t{nil},t+i\times 8}\rule{Idx}{^\&}\]

\[\dfrac{
    \rho,\mu,\chi,v \gives a \tto \rho,\mu',\t{nil},v_a
}{\rho,\mu,\chi,v \gives \t{OP1}(\t{M\_ADDR}, \t{OP1}(\t{M\_DEREF}, a)) \tto \rho,\mu',\t{nil},v_a}\rule{Ptr}{^\&}\]

Dereferencing (added in \Cmp):
\[\dfrac{
    \rho,\mu,\chi,v \gives a\tto \rho,\mu',\t{nil},v_a \qquad v_a\in\dom\mu'
}{\rho,\mu,\chi,v \gives \t{OP1}(\t{M\_DEREF}, a) \tto \rho,\mu',\t{nil},\mu'(v_a)}\rule{Ptr}{}\]

When the operand raises a non-\t{nil} flag:
\[\dfrac{
    \rho,\mu,\chi,v \gives e\tto \rho,\mu',\chi',v_e \qquad \chi'\ne\t{nil}
}{\rho,\mu,\chi,v \gives \t{OP1}(op, e) \tto \rho,\mu',\chi',v_e}\rule{Op1}{^\chi}\]

\subsection{Binary operators}
Multiplication (same as \Cmm):
\[\dfrac{\begin{array}{c}
    \rho,\mu,\chi,v \gives e_2\tto \rho,\mu_2,\chi_2,v_2 \\
    \rho,\mu_2,\chi_2,v_2 \gives e_1\tto \rho,\mu',\t{nil},v_1
\end{array}}{\rho,\mu,\chi,v \gives \t{OP2}(\t{S\_MUL}, e_1, e_2) \tto \rho,\mu',\t{nil},v_1\times v_2}\rule{Mul}{}\]

Addition (same as \Cmm):
\[\dfrac{\begin{array}{c}
    \rho,\mu,\chi,v \gives e_2\tto \rho,\mu_2,\chi_2,v_2 \\
    \rho,\mu_2,\chi_2,v_2 \gives e_1\tto \rho,\mu',\t{nil},v_1
\end{array}}{\rho,\mu,\chi,v \gives \t{OP2}(\t{S\_ADD}, e_1, e_2) \tto \rho,\mu',\t{nil},v_1 + v_2}\rule{Add}{}\]

Subtraction (same as \Cmm):
\[\dfrac{\begin{array}{c}
    \rho,\mu,\chi,v \gives e_2\tto \rho,\mu_2,\chi_2,v_2 \\
    \rho,\mu_2,\chi_2,v_2 \gives e_1\tto \rho,\mu',\t{nil},v_1
\end{array}}{\rho,\mu,\chi,v \gives \t{OP2}(\t{S\_SUB}, e_1, e_2) \tto \rho,\mu',\t{nil},v_1 - v_2}\rule{Sub}{}\]

Division and remainder (same as \Cmm):
\[\dfrac{\begin{array}{c}
    \rho,\mu,\chi,v \gives e_2\tto \rho,\mu_2,\chi_2,v_2 \qquad v_2\ne 0\\
    \rho,\mu_2,\chi_2,v_2 \gives e_1\tto \rho,\mu',\t{nil},v_1
\end{array}}{\rho,\mu,\chi,v \gives \t{OP2}(\t{S\_DIV}, e_1, e_2) \tto \rho,\mu',\t{nil},v_1\div v_2}\rule{Div}{}\]
\[\dfrac{\begin{array}{c}
    \rho,\mu,\chi,v \gives e_2\tto \rho,\mu_2,\chi_2,v_2 \qquad v_2\ne 0\\
    \rho,\mu_2,\chi_2,v_2 \gives e_1\tto \rho,\mu',\t{nil},v_1
\end{array}}{\rho,\mu,\chi,v \gives \t{OP2}(\t{S\_MOD}, e_1, e_2) \tto \rho,\mu',\t{nil},v_1\mod v_2}\rule{Mod}{}\]

Shifts (added in \Cmp):
\[\dfrac{\begin{array}{c}
    \rho,\mu,\chi,v \gives e_2\tto \rho,\mu_2,\chi_2,v_2 \\
    \rho,\mu_2,\chi_2,v_2 \gives e_1\tto \rho,\mu',\t{nil},v_1
\end{array}}{\rho,\mu,\chi,v \gives \t{OP2}(\t{S\_SHL}, e_1, e_2) \tto \rho,\mu',\t{nil},v_1\times 2^{v_2}}\rule{Shl}{}\]
\[\dfrac{\begin{array}{c}
    \rho,\mu,\chi,v \gives e_2\tto \rho,\mu_2,\chi_2,v_2 \\
    \rho,\mu_2,\chi_2,v_2 \gives e_1\tto \rho,\mu',\t{nil},v_1
\end{array}}{\rho,\mu,\chi,v \gives \t{OP2}(\t{S\_SHR}, e_1, e_2) \tto \rho,\mu',\t{nil},v_1\div 2^{v_2}} \rule{Shr}{}\]

Let \(\dec: \{ \bot, \top \}^{64} \to \ZZ_{64}\) the function
\[(b_0, \cdots, b_{63}) \mapsto \sum_{i=0}^{63} (1\t{ if } b_i\t{ else }0)\times 2^i\]
and \(\bin = \dec^{-1}\).

We can now define bitwise operators as follows (added in \Cmp).\\
\[\dfrac{\begin{array}{c}
    \rho,\mu,\chi,v \gives e_2\tto \rho,\mu_2,\chi_2,v_2 \qquad \rho,\mu_2,\chi_2,v_2\gives e_1\tto \rho,\mu',\t{nil},v_1 \\
    (b_0^2,\cdots,b_{63}^2) = \bin(v_2) \qquad (b_0^1,\cdots,b_{63}^1) = \bin(v_1)
\end{array}}{\rho,\mu,\chi,v \gives \t{OP2}(\t{S\_AND}, e_1, e_2) \tto \rho,\mu',\t{nil},\dec(b_0^1 \wedge b_0^2,\cdots, b_{63}^1 \wedge b_{63}^2),\mu''} \rule{And}{}\]
\[\dfrac{\begin{array}{c}
    \rho,\mu,\chi,v \gives e_2\tto \rho,\mu_2,\chi_2,v_2 \qquad \rho,\mu_2,\chi_2,v_2\gives e_1\tto \rho,\mu',\t{nil},v_1 \\
    (b_0^2,\cdots,b_{63}^2) = \bin(v_2) \qquad (b_0^1,\cdots,b_{63}^1) = \bin(v_1)
\end{array}}{\rho,\mu,\chi,v \gives \t{OP2}(\t{S\_OR}, e_1, e_2) \tto \rho,\mu',\t{nil},\dec(b_0^1 \vee b_0^2,\cdots, b_{63}^1 \vee b_{63}^2),\mu''} \rule{Ior}{}\]
\[\dfrac{\begin{array}{c}
    \rho,\mu,\chi,v \gives e_2\tto \rho,\mu_2,\chi_2,v_2 \qquad \rho,\mu_2,\chi_2,v_2\gives e_1\tto \rho,\mu',\t{nil},v_1 \\
    (b_0^2,\cdots,b_{63}^2) = \bin(v_2) \qquad (b_0^1,\cdots,b_{63}^1) = \bin(v_1)
\end{array}}{\rho,\mu,\chi,v \gives \t{OP2}(\t{S\_XOR}, e_1, e_2) \tto \rho,\mu',\t{nil},\dec(b_0^1 \oplus b_0^2,\cdots, b_{63}^1 \oplus b_{63}^2),\mu''} \rule{Xor}{}\]

When one of the operands raises a non-\t{nil} flag:
\[\dfrac{\begin{array}{c}
    \rho,\mu,\chi,v \gives e_2\tto \rho,\mu_2,\chi_2,v_2 \\
    \rho,\mu_2,\chi_2,v_2 \gives e_1\tto \rho,\mu',\chi',v_1 \\
    \chi'\ne\t{nil}
\end{array}}{\rho,\mu,\chi,v \gives \t{OP2}(op, e_1, e_2) \tto \rho,\mu',\chi',v_1} \rule{Op2}{^\chi}\]

\subsection{Comparisons}
All are the same as in \Cmm.
\[\dfrac{\begin{array}{c}
    \rho,\mu,\chi,v \gives e_2\tto \rho,\mu_2,\chi_2,v_2 \\
    \rho,\mu_2,\chi_2,v \gives e_1\tto \rho,\mu',\t{nil},v_1 \\
    \qquad v_1 = v_2
\end{array}}{\rho,\mu,\t{nil},v \gives \t{CMP}(\t{C\_EQ}, e_1, e_2) \tto \rho,\mu',\t{nil},1}\rule{Eq}{^\top}\]
\[\dfrac{\begin{array}{c}
    \rho,\mu,\chi,v \gives e_2\tto \rho,\mu_2,\chi_2,v_2 \\
    \rho,\mu_2,\chi_2,v \gives e_1\tto \rho,\mu',\t{nil},v_1 \\
    \qquad v_1 < v_2
\end{array}}{\rho,\mu,\t{nil},v \gives \t{CMP}(\t{C\_LT}, e_1, e_2) \tto \rho,\mu',\t{nil},1}\rule{Lt}{^\top}\]
\[\dfrac{\begin{array}{c}
    \rho,\mu,\chi,v \gives e_2\tto \rho,\mu_2,\chi_2,v_2 \\
    \rho,\mu_2,\chi_2,v \gives e_1\tto \rho,\mu',\t{nil},v_1 \\
    \qquad v_1 \leq v_2
\end{array}}{\rho,\mu,\t{nil},v \gives \t{CMP}(\t{C\_LE}, e_1, e_2) \tto \rho,\mu',\t{nil},1}\rule{Le}{^\top}\]
\[\dfrac{\begin{array}{c}
    \rho,\mu,\chi,v \gives e_2\tto \rho,\mu_2,\chi_2,v_2 \\
    \rho,\mu_2,\chi_2,v \gives e_1\tto \rho,\mu',\t{nil},v_1 \\
    \qquad v_1 \ne v_2
\end{array}}{\rho,\mu,\t{nil},v \gives \t{CMP}(\t{C\_EQ}, e_1, e_2) \tto \rho,\mu',\t{nil},0}\rule{Eq}{^\bot}\]
\[\dfrac{\begin{array}{c}
    \rho,\mu,\chi,v \gives e_2\tto \rho,\mu_2,\chi_2,v_2 \\
    \rho,\mu_2,\chi_2,v \gives e_1\tto \rho,\mu',\t{nil},v_1 \\
    \qquad v_1 \not< v_2
\end{array}}{\rho,\mu,\t{nil},v \gives \t{CMP}(\t{C\_LT}, e_1, e_2) \tto \rho,\mu',\t{nil},0}\rule{Lt}{^\bot}\]
\[\dfrac{\begin{array}{c}
    \rho,\mu,\chi,v \gives e_2\tto \rho,\mu_2,\chi_2,v_2 \\
    \rho,\mu_2,\chi_2,v \gives e_1\tto \rho,\mu',\t{nil},v_1 \\
    \qquad v_1 \not\leq v_2
\end{array}}{\rho,\mu,\t{nil},v \gives \t{CMP}(\t{C\_LE}, e_1, e_2) \tto \rho,\mu',\t{nil},0}\rule{Le}{^\bot}\]

For optimisation purposes mostly, the comparison operators \(\t{C\_NE}\), \(\t{C\_GT}\), \(\t{C\_GE}\) may be introduced by the compiler (not by the parser, however).\\
They are defined as
\[\dfrac{\begin{array}{c}
    \rho,\mu,\chi,v \gives e_2\tto \rho,\mu_2,\chi_2,v_2 \\
    \rho,\mu_2,\chi_2,v_2 \gives e_1\tto \rho,\mu',\t{nil},v_1 \\
    \qquad v_1 = v_2
\end{array}}{\rho,\mu,\t{nil},v \gives \t{CMP}(\t{C\_NE}, e_1, e_2) \tto \rho,\mu',\t{nil},0}\rule{Ne}{^\bot}\]
\[\dfrac{\begin{array}{c}
    \rho,\mu,\chi,v \gives e_2\tto \rho,\mu_2,\chi_2,v_2 \\
    \rho,\mu_2,\chi_2,v_2 \gives e_1\tto \rho,\mu',\t{nil},v_1 \\
    \qquad v_1 \leq v_2
\end{array}}{\rho,\mu,\t{nil},v \gives \t{CMP}(\t{C\_GT}, e_1, e_2) \tto \rho,\mu',\t{nil},0}\rule{Gt}{^\bot}\]
\[\dfrac{\begin{array}{c}
    \rho,\mu,\chi,v \gives e_2\tto \rho,\mu_2,\chi_2,v_2 \\
    \rho,\mu_2,\chi_2,v_2 \gives e_1\tto \rho,\mu',\t{nil},v_1 \\
    \qquad v_1 < v_2
\end{array}}{\rho,\mu,\t{nil},v \gives \t{CMP}(\t{C\_GE}, e_1, e_2) \tto \rho,\mu',\t{nil},0}\rule{Ge}{^\bot}\]
\[\dfrac{\begin{array}{c}
    \rho,\mu,\chi,v \gives e_2\tto \rho,\mu_2,\chi_2,v_2 \\
    \rho,\mu_2,\chi_2,v_2 \gives e_1\tto \rho,\mu',\t{nil},v_1 \\
    \qquad v_1 \ne v_2
\end{array}}{\rho,\mu,\t{nil},v \gives \t{CMP}(\t{C\_NE}, e_1, e_2) \tto \rho,\mu',\t{nil},1}\rule{Ne}{^\top}\]
\[\dfrac{\begin{array}{c}
    \rho,\mu,\chi,v \gives e_2\tto \rho,\mu_2,\chi_2,v_2 \\
    \rho,\mu_2,\chi_2,v_2 \gives e_1\tto \rho,\mu',\t{nil},v_1 \\
    \qquad v_1 \not\leq v_2
\end{array}}{\rho,\mu,\t{nil},v \gives \t{CMP}(\t{C\_GT}, e_1, e_2) \tto \rho,\mu',\t{nil},1}\rule{Gt}{^\top}\]
\[\dfrac{\begin{array}{c}
    \rho,\mu,\chi,v \gives e_2\tto \rho,\mu_2,\chi_2,v_2 \\
    \rho,\mu_2,\chi_2,v_2 \gives e_1\tto \rho,\mu',\t{nil},v_1 \\
    \qquad v_1 \not< v_2
\end{array}}{\rho,\mu,\t{nil},v \gives \t{CMP}(\t{C\_GE}, e_1, e_2) \tto \rho,\mu',\t{nil},1}\rule{Ge}{^\top}\]

When one of the operands raises a non-\t{nil} flag:
\[\dfrac{\begin{array}{c}
    \rho,\mu,\chi,v \gives e_2\tto \rho,\mu_2,\chi_2,v_2 \\
    \rho,\mu_2,\chi_2,v_2 \gives e_1\tto \rho,\mu',\chi',v_1 \\
    \chi'\ne\t{nil}
\end{array}}{\rho,\mu,\chi,v \gives \t{CMP}(op, e_1, e_2) \tto \rho,\mu',\chi',v_1} \rule{Cmp}{^\chi}\]


\subsection{Assignments}

\[\dfrac{\begin{array}{c}
    \rho,\mu,\chi,v \gives e\tto \rho,\mu',\t{nil},v_e \\
    x\in\dom\rho \qquad \rho(x)\in\dom\mu'
\end{array}}{\rho,\mu,\chi,v \gives \t{SET\_VAR}(x, e) \tto \rho,\mu'[\rho(x)\mapsto v_e],\t{nil},v_e}\rule{Var}{^\gets}\]
\[\dfrac{\begin{array}{c}
    \rho,\mu,\chi,v \gives e\tto \rho,\mu',\chi',v_e \qquad \chi'\ne\t{nil}
\end{array}}{\rho,\mu,\chi,v \gives \t{SET\_VAR}(x, e) \tto \rho,\mu',\chi',v_e}\rule{Var}{^{\gets\chi}}\]

\[\dfrac{\begin{array}{c}
    \rho,\mu,\chi,v \gives e\tto \rho,\mu_e,\chi_e,v_e \\
    \rho,\mu_e,\chi_e,v_e \gives i\tto \rho,\mu',\t{nil},v_i \\
    \rho(x)\in\dom\mu'
\end{array}}{\rho,\mu,\chi,v \gives \t{SET\_ARRAY}(x, i, e) \tto \rho,\mu'[\rho(x)+v_i\times 8 \mapsto v_e],\t{nil},v_e}\rule{Idx}{^\gets}\]
\[\dfrac{\begin{array}{c}
    \rho,\mu,\chi,v \gives e\tto \rho,\mu_e,\chi_e,v_e \\
    \rho,\mu_e,\chi_e,v_e \gives i\tto \rho,\mu',\chi',v_i \qquad \chi'\ne\t{nil}
\end{array}}{\rho,\mu,\chi,v \gives \t{SET\_ARRAY}(x, i, e) \tto \rho,\mu',\chi',v_i}\rule{Idx}{^{\gets\chi}}\]

\[\dfrac{\begin{array}{c}
    \rho,\mu,\chi,v \gives e \tto \rho,\mu_e,\chi_e,v_e \\
    \rho,\mu_e,\chi_e,v_e \gives a \tto \rho,\mu',\t{nil},v_a
\end{array}}{\rho,\mu,\chi,v \gives \t{SET\_DEREF}(a, e) \tto \rho,\mu'[v_a\mapsto v_e],\t{nil},v_e}\rule{Ptr}{^\gets}\]
\[\dfrac{\begin{array}{c}
    \rho,\mu,\chi,v \gives e \tto \rho,\mu_e,\chi_e,v_e \\
    \rho,\mu_e,\chi_e,v_e \gives a \tto \rho,\mu',\chi',v_a \qquad \chi'\ne\t{nil}
\end{array}}{\rho,\mu,\chi,v \gives \t{SET\_DEREF}(a, e) \tto \rho,\mu',\chi',v_a}\rule{Ptr}{^{\gets\chi}}\]

\end{document}
